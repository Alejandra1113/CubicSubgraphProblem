\documentclass{article}
\usepackage[spanish]{babel}
\usepackage[utf8]{inputenc}
\usepackage{graphicx}
\usepackage{amsmath}
\usepackage{amssymb}
\usepackage{amsthm}

\begin{document}

\title{\textbf{Proyecto \#3 DAA:} La venganza de Alejandra}
\author{L\'azaro Daniel Gonz\'alez Mart\'inez y Alejandra Monz\'on Pe\~na}

\date{}
\maketitle

\section*{Desccripci\'on del problema}
Dado un grafo $G$, eliminar aristas hasta que todos sus v\'ertices sean de grado $3$ o $0$, sin eliminarlas todas.\\ 

Esto es equivalente a determinar si un grafo $G$ tiene un subgrafo $G^{'}$ regular de grado $3$. \\

Los grafos regulares de grado $3$ se conocen tambi\'en como \textit{grafos c\'ubicos}, por lo que usaremos ambos t\'erminos indistintamente.

\section*{Reducci\'on de 3-coloreable a subgrafo 3-regular}
Primero, demostrar que el problema de subgrafo c\'ubico (CSP [cubic subgraph problem]) cumple $ CSP \subseteq NP$ es sencillo, ya que si tuvi\'esemos un grafo $G$ y conoci\'eramos 
que aristas de \'el componen el subgrafo c\'ubico, basta con recorrer el conjunto $E$ de aristas a conservar del grafo $G$ como parte del subgrafo c\'ubico 
y por cada par de v\'ertices sobre los que incide aumentar en $1$ su degree, finalmente basta con recorrer la lista de v\'ertices de $G$ y comprobar que todos tengan degree 
$0$ o  $3$(y al menos se tuviera una arista), procedimiento cuyo costo ser\'ia $O(|V| + |E|)$, como existe una forma polin\'omica de comprobar si una instancia del problema realmente cumple con 
la caracter\'istica deseada. Queda entonces demostrar que el problema $CSP = NP$.\\

Para demostrar que el problema de determinar si un grafo tine un subgrafo c\'ubico es NP-completo, es necesario hacer una reducci\'on de un problema que sea NP-completo 
a este problema; para esto utilizaremos el problema $3$-coloreable, ya que el problema de la 3-coloraci\'on de un grafo se demostr\'o NP-completo por Karps.\\

Definamos t\'erminos a utilizar en la demostraci\'on: 
\begin{itemize}
    \item[$\ast $] $G$: grafo de entrada del problema 3-coloreable.
    \item[$\ast $] $V(G)~ = ~\{v_1,~ v_2,~ ... ~, ~v_n\}$: v\'ertices de $G$.
    \item[$\ast $] $E(G)~ = ~\{e_1, ~e_2,~ ...~ ,~ e_m\}$: aristas de $G$.
    \item[$\ast $] $<u,v>$: Arista que conecta los v\'ertices $u$ y $v$
    \item[$\ast $] $d(v_i)$: Grado del v\'ertice $v_i$.
    \item[$\ast $] $\Delta(G)$: Mayor degree de los v\'ertices del grafo.
    \item[$\ast $] $K^{'}_n$: Grafo completo de $n$ v\'ertices al que se le quita una arista(tiene todos los v\'ertices con degree n-1, excepto 2 v\'ertices de degree n-2).
    \item[$\ast $] $3$-partici\'on de $V(G)$: Conjuntos $V_1$, $V_2$ y $V_3$, tales que $\bigcup_{i=1}^{3} ~V_i~ = ~V(G)$  y $V_i ~\cap~ V_j~ =~ \emptyset$ para todo $i~ \neq~ j$.
\end{itemize}

\textbf{Transformaci\'on de la entrada:}\\

A partir del grafo $G$ se contruye un nuevo grafo $G'$ de la siguiente forma:\\ 

1 - Por cada $v_i$ de $V(G)$ se crean $3$ ciclos $C_i^1$, $C_i^2$ y $C_i^3$, cada uno de longitud igual a $2 ~*~ d(v_i)~ + ~1$. Denotemos 
los v\'ertces de cada ciclo $C_i^h$ como $c_{ij}^h$, $1~ \leq~i~\leq ~n$, $1~ \leq~d~\leq ~2 ~*~ d(v_i)~ + ~1$, $1~ \leq~h~\leq ~3$.\\ 

2 - Por cada $e_j$ de $E(G)$ se crean $3$ subgrafos $D_{j}^1$, $D_{j}^2$ y $D_{j}^3$ en $G'$, donde cada $D_{j}^h$, $1~ \leq~h~\leq ~3$,  es un $K^{'}_4$, a los dos v\'ertices 
de $D_{j}^h$ con grado $2$ los denotaremos como $x_{j}^h$ y $y_{j}^h$.\\

3 - Por cada $e_j$, sean $v_s$ y $v_t$ los v\'ertices sobre los que incide en $G$, de los ciclos $C_s^h$ ($C_t^h$) se toman $2$ v\'ertices 
$c_{s\alpha}^h$ y $c_{s\beta}^h$ ($c_{s\alpha}^h$ y $c_{s\beta}^h$) que aun tengan degree $2$; por cada $1~\leq ~h ~\leq~ 3$ se agragan a $G'$ las aristas: $<c_{s\alpha}^h, x_{j}^h>$, $<c_{s\beta}^h, y_{j}^h>$, 
$<c_{t\alpha}^h, x_{j}^h>$ y $<c_{t\beta}^h, y_{j}^h>$.\\ 

Una vez consideradas todas las aristas en el paso (3), se tiene que para cada ciclo $C_i^h$ ( $1~\leq ~i ~\leq~ n$, $1~\leq ~h ~\leq~ 3$ )
solo queda un v\'ertice con degree $2$. Nombremos dichos v\'ertices como $w_i^h$.\\ 

4 - Por cada $1~\leq ~i ~\leq~ n$, se construye un subgrafo $U_i$, que es un $K^{'}_4$ m\'as u v\'ertice al que denominaremos $u_{i}$, los v\'ertices de grado $2$ del $K^{'}_4$
los denominaremos $x_{i}$ y $y_{i}$, el v\'ertice $u_i$ se une a los restantes del  $K^{'}_4$ mediante una arista $<x_{i},u_i>$.\\ 

Se toman todos los grafos $U_i$ y se agregan a $G'$, junto con las aristas: $<u_i, w_i^1>$, $<y_i, w_i^2>$ y $<y_i, w_i^3>$. \\

5 - Agregar el ciclo $C'$ de longitud $2~*~n$ a $G'$, conformado por los v\'ertices: $\{ a_{11}, ... , a_{1n}, a_{21}, ... , a_{2n} \}$ y agregar las aristas 
$<a_{pi}, u_{i}>$ para $p = 1,2$.\\

Construido $G'$, queda demostrar que en $G$ es $3$-coloreable si y solo si $G'$ tiene un subgrafo c\'ubico. \\ 

($\Rightarrow$) Sea $G$ un grafo $3$-coloreable, y una $3$-partici\'on de $V(G)$ tal que en cada conjunto $V_i$ queden solo v\'ertices de una mismo 
color de $G$, entonces existe un grafo $G'[H]$, subgrafo inducido de $G'$ que es $3$-regular, ya que siempre podemos tomar los v\'ertices de la siguiente forma:\\

\begin{enumerate}
    \item Todos los v\'ertices $a_{ij}$ est\'an en $V(H)$
    \item Todos los v\'ertices $u_{i}$ est\'an en $V(H)$
    \item Si $v_i$ de $G$ est\'a en el conjunto $V_c$ de la tricoloraci\'on, entonces los v\'ertices $c_{ij}$ del ciclo $C_i^c$ est\'an en $H$.
    \item Si $c ~\neq ~1$ para el conjunto $V_c$ al que pertenece $v_i$, entonces los v\'ertices de $U_i$ pertenecen a $H$.
    \item Si la arista $e_j$, es adyacente al v\'ertice $v_i$ que est\'a en el conjunto $V_c$, entonces los v\'ertices de $D_j^c$ adyacente a $C_i^c$ est\'an en $H$.
\end{enumerate}
    El subgrafo $G'[H]$ existe para cualquier $3$-partici\'on de $G$, y cuando la $3$-partici\'on corresponde con una 
    coloraci\'on se puede comprobar que todos los v\'ertices en $G'[H]$ tienen grado $3$, en principio en $G'$ todos los v\'ertices son de grado $3$ excepto los 
    $x_{jp}^c$ y $y_{jp}^c$ de los $D_{jp}^c$ que son de grado $4$ pero en el subgrafo se hace una construcci\'on a partir de una coloraci\'on y en $G'$ los $D_{jp}^c$ grafos reemplazad aristas, 
    entonces los v\'ertices $x_{jp}^c$ y $y_{jp}^c$ est\'an conctados a v\'ertices de c\'irculos que no pertenecer\'an ambos a $H$, de donde obligatoriamente quedan en grado $3$. De igual modo ocurre con los v\'ertices $u_i$ y $y_i$ de los $U_i$, quienes tienen 
    grado $4$ cada uno en $G$, pero como $H$ lo formamos considerando la $3$-partici\'on y cada v\'ertice puede estar s\'olo en uno de los $3$ conjuntos, entonces de las aristas que inciden en $u_i$ solo se quedan las $2$ que lo conectan al ciclo $C'$ y la que se corresponde a si $v_i$ 
    est\'a en el conjunto $V_1$ o no, y en los casos en los que se toma alg\'un $y_i$ como parte del conjunto $H$, en \'el solo permanecen las dos aristas que lo conectan al resto del $U_i$ y solo una de las que indica si el v\'ertice est\'a en el conjunto $V_2$ o en el $V_3$ de la $3$-partici\'on. 
    Por tanto en $G'[H]$, todos los v\'ertices tienen grado $3$.\\ 

($\Leftarrow$) Si $G'$ contiene un subgrafo $3$-regular $G'[H]$, entonces sobre $H$ se cumplen las siguientes propiedades:\\ 

\begin{enumerate}
    \item Todos los v\'ertices $a_{pi}$ y $u_i$ pertenecen a $H$.
    \item Por cada $i$, $1~ \leq~i~\leq ~n$, exactamente uno de los ciclos $C_i^{h}$ est\'a en $H$.
    \item Por cada $i$, $1~ \leq~i~\leq ~n$, si $C_i^{h}$ est\'a en $H$, entonces ning\'un otro $C_j^{h}$ para toda $j$ tal que $<i,j> \in E(G)$.
\end{enumerate}

Por la proposici\'on (2) implica que el subgrafo $G'[H]$ es una $3$-partici\'on de los v\'ertices de $G'$ tal que el v\'ertice
$v_i$ est\'a en la partici\'on $c$ si $C_i^c \in H$. La proposici\'on (3) asegura que v\'ertices adyacentes est\'en en diferentes conjuntos en la partici\'on, de donde
si $G'$ tiene $G'[H]$ como subgrafo c\'ubico, entonces $G$ es $3$-coloreable. $\Box$  \\ 

\subsection*{Reducci\'on implementada}
Para poder comparar resultados en la pr\'actica 
\subsection*{Soluci\'on Backtrack}
La soluci\'on exacta al problema del subgrafo $3$-regular fue programada haciendo uso de un algoritmo de Backtrack en el que para cada una de las aristas del grafo se tienen 
dos posibles opciones, esta forma parte del subgrafo $3$-regular o no, por tanto se genera un \'arbol binario en el que cada rama corresponde a la selecci\'on o no de una arista
para ser parte del subgrafo. 

\subsection*{Cuestiones sobre subgrafos 3-regulares}
\textbf{Teorema}: Para todo $p$ primo, cualquier grafo $G=(V,E)$ tal que el promedio de los grados de sus v\'ertices sea mayor que 
$2p-2$ y el $\Delta(G)$ sea a lo sumo $2p -1$, tiene un subgrafo $p$-regular.\\ 

\textbf{Demostraci\'on}: Sean los valores binarios $a_{i,j}$, tales que $a_{i,j} = 1$ si la arista $e_{j}$ incide en el v\'ertice $v_i$, $a_{i,j} = 0$ en otro caso.
Considerando el polinomio de variables $x_e,~ e\in E(G)$:

$$P(x_e) = \prod_{v \in V(G)} \left( 1 - \left(\sum_{e \in E(G)} a_{ve} x_e\right)^{(p-1)}\right) - \prod_{e \in E(G)} (1 - x_e)$$ 

Como se tiene que el grado promedio es mayor que $2p-2$, entonces: 

$$\frac{2|E|}{|V|} > 2p-2 $$
$$|E| > (p-1)|V|$$

Por tanto, se tiene que el grado de $P$ es $|E| = \sum_{e \in E} t_e$, donde $t_e = 1$ para $e \in E$, luego se tiene que el 
coeficiente principal del polinomio es $- \prod_{e \in E}  - x_e^{t_e} = - \prod_{e \in E} - x_e = (-1)^{|E| + 1} \neq 0$. Sean los 
conjuntos de soluci\'on $S_e = {0,1}$ para toda $e \in E$, como $|S_e| > t_e$ para toda $e \in E$, entonces por el Corolario del \textit{Teorema Combinatoria Nullstellensatz}, se tiene 
que existe $s_e \in S_e$, tales que $P((s_1,...s_m))  \neq 0$. Dicho vector no es el vector $0$, ya que $P(0) = \prod_{v \in V(G)}  1 - \prod_{e \in E(G)} 1 = 1 - 1 = 0$. De aqui se tiene que entonces 
como ag\'un $s_e$ debe ser igual a $1$, entonces $\prod_{e \in E(G)} (1 - s_e) = 0$, de donde para cada $v \in V$ se tiene que cumplir que 
$\sum_{e \in E(G)} a_{ve} x_e \equiv 0$ (mod $p$), puesto que de lo contario por el \textit{Peque\~no Teorema de Fermat} se tiene que $(\sum_{e \in E(G)} a_{ve} s_e)^(p-1) \equiv 1$ (mod $p$), de donde 
$P((s_1,...,s_m)) = 0 $. Por tanto en el subgrafo conformado por las aristas $e$ tales que $s_e = 1$, se tiene que todos los v\'ertices extremos de estas aristas tendr\'ian grado $p$, puesto a que para ellos
$(\sum_{e \in E(G)} a_{ve} s_e) > 0$, divisible entre $p$ y menor que $2p-1$. $\Box $


\end{document}